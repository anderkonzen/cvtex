%-----------------------------------------------------------------------
% Anderson Konzen (anderson.konzen at gmail)
% Template of a CV typeset in XeTeX
% 
% URL: https://github.com/anderkonzen/xetexcv
% DISCLAIMER: This template is provided for free and without any
%             guarantee that it will correctly compile on your
%             system if you have a non-standard configuration.
%
% (This work is a fork from https://github.com/dartar/cvtex)
%
% This modified version is licensed under the same Creative Commons
% Attribution-ShareAlike license:
% http://creativecommons.org/licenses/by-sa/3.0/.
%-----------------------------------------------------------------------

%!TEX TS-program = xelatex
%!TEX encoding = UTF-8 Unicode

\documentclass[10pt,a4paper]{article}

\def\myname{Anderson Konzen}
\def\myemail{anderson.konzen@gmail.com}
\def\myskype{andersonkonzen}
\def\mycellphone{+55 51 9196 0949}
\def\myaddress{Av. Cauduro, 57/301\\Porto Alegre, \textsc{rs} - 90035-030\\Brazil}


%% Loads the fontspec package, so we can easily select system fonts
%\usepackage{fontspec}
%% It implements some odds-and-ends features and improved functionality for broken or sub-standard LaTeX methods when using the XeTeX format. Already loads the fontspec package.
\usepackage{xltxtra}
%% Loads xcolor package, so we can use color names instead of rgb values
\usepackage{xcolor}

%-----------------------------------------------------------------------
% Document Layout
%-----------------------------------------------------------------------
%% Loads geometry package, so we can easily change page setup
\usepackage{geometry}
\geometry{a4paper,         %% Specifies the paper size
  textwidth=5.5in,         %% Specifies the width of body (the text area)
  textheight=9.0in,        %% Specifies the height of body (the text area)
  marginparsep=7pt,        %% Modifies separation between body and marginal notes
  marginparwidth=.6in}     %% Modifies width of the marginal notes
\setlength\parindent{0in}  %% Amount of indentation at the first line of a paragraph.

%-----------------------------------------------------------------------
% Fonts
%-----------------------------------------------------------------------
\setmainfont[Mapping={tex-text},Numbers={OldStyle},Ligatures={Common}]{Linux Libertine O}
\setsansfont[Mapping={tex-text},Color=AA0000]{Linux Biolinum O}
\setmonofont[Mapping={tex-text},Scale=0.8]{Inconsolata}

%-----------------------------------------------------------------------
% Custom Commands
%-----------------------------------------------------------------------
\chardef\&="E050  % Redefine '&' character
\newcommand{\html}[1]{\href{#1}{\scriptsize\textsc{[html]}}}
\newcommand{\pdf}[1]{\href{#1}{\scriptsize\textsc{[pdf]}}}
\newcommand{\doi}[1]{\href{#1}{\scriptsize\textsc{[doi]}}}
%% Configure margin years
\usepackage{marginnote}
\newcommand{\years}[1]{\marginnote{#1}}
\renewcommand*{\raggedleftmarginnote}{}  %% Define margin note alignment (in this case, justified text at the left margin)
%\setlength{\marginparsep}{7pt}  %% Already defined in geometry
\reversemarginpar  %% Margin notes in left side of the page

%-----------------------------------------------------------------------
% Headings
%-----------------------------------------------------------------------
%% Provides a set of commands for changing the font used for the various sectional headings
\usepackage{sectsty}
%% Provides various types of underlining that can stretch between words and be broken across lines
\usepackage[
  normalem  %% \em and \emph still produce normal italics
]{ulem}
%% Change font size of section headers
\sectionfont{\color[HTML]{AA0000}\mdseries\upshape\Large}
\subsectionfont{\mdseries\scshape\normalsize} 
\subsubsectionfont{\mdseries\upshape\large} 

%-----------------------------------------------------------------------
% PDF Setup
%-----------------------------------------------------------------------
\usepackage[
  xetex,            %% Use XeTeX backend
  colorlinks=true,  %% Colors the text of links and anchors
  breaklinks=true,  %% Allow links to break over lines
  pdftitle={{\myname} - Curriculum Vitae},
  pdfauthor={\myname}
]{hyperref}  
\hypersetup{linkcolor=blue,filecolor=black,urlcolor=blue}


%-----------------------------------------------------------------------
%-----------------------------------------------------------------------
% DOCUMENT
%-----------------------------------------------------------------------
\begin{document}
{\LARGE \myname}\\
\hrule
\vspace{0.2in}
{\large phone: \mycellphone}\\[.05cm]
{\large email: \href{mailto:\myemail}{\texttt{\myemail}}}\\[.05cm]
{\large skype: \myskype}\\[.2cm]
\myaddress
%\vfill
\vspace{0.5in}

%\hrule
\section*{Professional Experience}

\subsection*{Hewlett-Packard | {\footnotesize{porto alegre, rs, brazil}}}

\years{2010- \ldots}Software Engineer\\
Design and development of embedded software solution for a new product family of Network Scanners, Single Function Printers and MFPs of HP. The development is focused on the UI layer, using C\#, .NET and .NET Compact frameworks; HTML5 and WebServices.\\
Due to the geographical distribution of the teams, I work with colleagues from the U.S. and India on a daily basis.

\subsection*{Actia do Brasil | {\footnotesize{porto alegre, rs, brazil}}}

\years{2007-2010}Project Coordinator\\
During this time I worked as a coordinator for the projects of the Automotive Diagnostic BU, where I could improve my management skills.
\begin{itemize}
\item[-] Coordination of distributed teams engaged in software development for the company's automotive diagnostic business unit. Main objectives included reaching goals of performance, client satisfaction and delivery of artifacts to clients.
\item[-] Management tasks were also improved on a daily basis: management of timeline, risks, people and communication between teams.
\item[-] Worked as a developer in critical situations, helping with development, training and application of best practices of software engineering.
\item[-] Mapping development process within the scope of the business unit.
\item[-] Constant contact with teams of other subsidiaries to cooperate with development, mainly located in France and Germany.
\end{itemize}

\years{2005-2007}Software Developer\\
Development of automotive diagnostic systems:
\begin{itemize}
\item[-] A web-based system was developed for Fiat do Brasil to allow FIAT engineers to remotely diagnose and repair electronic control units. The client side is a windows desktop application developed in C++/Visual C++ and the server side was developed in Java/JSP/Oracle/JBoss.
\item[-] Development and updates for the VCI (vehicle communication interface) firmware.
\item[-] C/C++/XML programming to add, update and fix protocols in Actia's solution for automotive diagnostics.
\item[-] Experience in automotive communication protocols.
\end{itemize}
Other projects included:
\begin{itemize}
\item[-] Development of digital video capture system for automotive surveillance using embedded Linux.
\item[-] Creation and support of installers for several Actia products using NSIS.
\end{itemize}

\subsection*{W3Haus | {\footnotesize{porto alegre, rs, brazil}}}

\years{2003-2005}Web Developer\\
Development of websites and intranet systems mainly using PHP/ASP/C\# and PostgreSQL/SQLServer.

%\hrule
\section*{Education}

\years{2010-\ldots}\textsc{MSc} in Electrical Engineering at Pontifícia Universidade Católica do Rio Grande do Sul - \textsc{pucrs}, Porto Alegre, \textsc{rs}, Brazil\\
{\em{Currently blocked.}}\\
\years{2006-2007}\textsc{MBA} in Business Administration at Fundação Getúlio Vargas - \textsc{fgv}, Porto Alegre, \textsc{rs}, Brazil\\
\years{1999-2003}\textsc{B.Eng} in Computer Engineering at Universidade Federal do Rio Grande do Sul - \textsc{ufrgs}, Porto Alegre, \textsc{rs}, Brazil

%\hrule
\section*{Languages}

\begin{itemize}
\item Portuguese (Native language)
\item English (Professional working proficiency)
\item French (Elementary proficiency)
\end{itemize}

%\hrule
\section*{References}

Available upon request.

%\vspace{1cm}
\vfill{}
%\hrulefill

\begin{center}
{\scriptsize  Last updated: \today\- $\bullet$\- 
% ---- PLEASE LEAVE THIS BACKLINK FOR ATTRIBUTION AS PER CC-LICENSE
Typeset in \href{http://nitens.org/taraborelli/cvtex}{\XeTeX}\\
% ---- FILL IN THE FULL URL TO YOUR CV HERE
\href{https://github.com/anderkonzen/cvtex}{https://github.com/anderkonzen/cvtex}}
\end{center}

\end{document}